\subsubsection*{CA-Struktur vorbereiten}

Für die CRL-Verwaltung benötigt OpenSSL eine fest definierte Verzeichnisstruktur.
Im Projektordner wird folgende Struktur angelegt:

\begin{verbatim}
pki/
|-- certs/
|-- crl/
|-- newcerts/
|-- private/
`-- index.txt
\end{verbatim}


Zusätzlich wird die Datei für die Seriennummer angelegt:

\begin{verbatim}
echo 1000 > serial
echo 1000 > crlnumber
touch index.txt
\end{verbatim}

\subsubsection*{OpenSSL-Konfigurationsdatei für CRLs}

Die Datei \texttt{openssl.cnf} wird so erweitert, dass CRLs erzeugt und verwaltet werden können:

\begin{verbatim}
[ ca ]
default_ca = CA_default

[ CA_default ]
dir             = ./pki
database        = $dir/index.txt
new_certs_dir   = $dir/newcerts
certificate     = $dir/certs/ca.cert.pem
private_key     = $dir/private/ca.key.pem
serial          = $dir/serial
crlnumber       = $dir/crlnumber
crl_dir         = $dir/crl
crl             = $crl_dir/ca.crl.pem
default_crl_days = 30
default_md      = sha256
\end{verbatim}

Damit OpenSSL später weiß, wohin es CRLs schreiben soll, wird außerdem ein CRL Distribution Point in der Konfiguration definiert:

\begin{verbatim}
[ v3_ca ]
crlDistributionPoints = URI:http://localhost/crl/ca.crl.pem
\end{verbatim}

\subsubsection*{Zertifikat widerrufen}

Ein ausgestelltes Zertifikat wird mit folgendem Kommando gesperrt:

\begin{verbatim}
openssl ca -config openssl.cnf -revoke pki/certs/server.cert.pem
\end{verbatim}

Dabei trägt OpenSSL automatisch einen neuen Eintrag in die Datei \texttt{index.txt} ein:

\begin{verbatim}
R    <Datum>   <Seriennummer>   unknown ...
\end{verbatim}

\subsubsection*{CRL erzeugen}

Nach dem Widerruf wird eine neue CRL erstellt:

\begin{verbatim}
openssl ca -config openssl.cnf -gencrl \
    -out pki/crl/ca.crl.pem
\end{verbatim}

Die Datei \texttt{ca.crl.pem} ist nun die gültige Sperrliste der CA.

\subsubsection*{CRL prüfen}

Die erzeugte CRL kann mit folgendem Befehl angezeigt werden:

\begin{verbatim}
openssl crl -in pki/crl/ca.crl.pem -text -noout
\end{verbatim}

Wichtige Ausgaben:

\begin{itemize}
    \item \textbf{Last Update} – Zeitpunkt der Erstellung
    \item \textbf{Next Update} – Gültigkeitsdauer der CRL
    \item \textbf{Revoked Certificates} – Liste der gesperrten Zertifikate
\end{itemize}

Beispielauszug:

\begin{verbatim}
Revoked Certificates:
    Serial Number: 1005
        Revocation Date: Mar  5 10:15:03 2025 GMT
\end{verbatim}

