\chapter{Certificate Revocation}

\section{CRL – Certificate Revocation Lists}

\subsection{Erstellung und Veröffentlichung einer CRL mit XCA und Nginx}
\input{chapters/06-Revokation/CRL_XCA_nginx/crlxca.tex}

\subsection{Erstellung und Veröffentlichung einer CRL mit OpenSSL und Nginx}
\subsubsection*{CA-Struktur vorbereiten}

Für die CRL-Verwaltung benötigt OpenSSL eine fest definierte Verzeichnisstruktur.
Im Projektordner wird folgende Struktur angelegt:

\begin{verbatim}
pki/
|-- certs/
|-- crl/
|-- newcerts/
|-- private/
`-- index.txt
\end{verbatim}


Zusätzlich wird die Datei für die Seriennummer angelegt:

\begin{verbatim}
echo 1000 > serial
echo 1000 > crlnumber
touch index.txt
\end{verbatim}

\subsubsection*{OpenSSL-Konfigurationsdatei für CRLs}

Die Datei \texttt{openssl.cnf} wird so erweitert, dass CRLs erzeugt und verwaltet werden können:

\begin{verbatim}
[ ca ]
default_ca = CA_default

[ CA_default ]
dir             = ./pki
database        = $dir/index.txt
new_certs_dir   = $dir/newcerts
certificate     = $dir/certs/ca.cert.pem
private_key     = $dir/private/ca.key.pem
serial          = $dir/serial
crlnumber       = $dir/crlnumber
crl_dir         = $dir/crl
crl             = $crl_dir/ca.crl.pem
default_crl_days = 30
default_md      = sha256
\end{verbatim}

Damit OpenSSL später weiß, wohin es CRLs schreiben soll, wird außerdem ein CRL Distribution Point in der Konfiguration definiert:

\begin{verbatim}
[ v3_ca ]
crlDistributionPoints = URI:http://localhost/crl/ca.crl.pem
\end{verbatim}

\subsubsection*{Zertifikat widerrufen}

Ein ausgestelltes Zertifikat wird mit folgendem Kommando gesperrt:

\begin{verbatim}
openssl ca -config openssl.cnf -revoke pki/certs/server.cert.pem
\end{verbatim}

Dabei trägt OpenSSL automatisch einen neuen Eintrag in die Datei \texttt{index.txt} ein:

\begin{verbatim}
R    <Datum>   <Seriennummer>   unknown ...
\end{verbatim}

\subsubsection*{CRL erzeugen}

Nach dem Widerruf wird eine neue CRL erstellt:

\begin{verbatim}
openssl ca -config openssl.cnf -gencrl \
    -out pki/crl/ca.crl.pem
\end{verbatim}

Die Datei \texttt{ca.crl.pem} ist nun die gültige Sperrliste der CA.

\subsubsection*{CRL prüfen}

Die erzeugte CRL kann mit folgendem Befehl angezeigt werden:

\begin{verbatim}
openssl crl -in pki/crl/ca.crl.pem -text -noout
\end{verbatim}

Wichtige Ausgaben:

\begin{itemize}
    \item \textbf{Last Update} – Zeitpunkt der Erstellung
    \item \textbf{Next Update} – Gültigkeitsdauer der CRL
    \item \textbf{Revoked Certificates} – Liste der gesperrten Zertifikate
\end{itemize}

Beispielauszug:

\begin{verbatim}
Revoked Certificates:
    Serial Number: 1005
        Revocation Date: Mar  5 10:15:03 2025 GMT
\end{verbatim}



\subsection{Verhalten moderner Browser bei CRLs}

\subsubsection*{Einordnung}

Nachdem im vorherigen Abschnitt gezeigt wurde, wie eine CRL mit XCA erzeugt und über einen Webserver wie Nginx bereitgestellt wird, stellt sich in der Praxis die Frage, welche Clients diese Informationen tatsächlich auswerten. Besonders relevant ist dabei das Verhalten moderner Webbrowser, da diese maßgeblich bestimmen, ob ein Zertifikatswiderruf im Alltag zuverlässig erkannt wird.

In realen Systemen zeigt sich jedoch, dass viele Browser die im Zertifikat hinterlegten Widerrufsinformationen – wie CRL Distribution Points oder OCSP-Server – entweder gar nicht oder nur eingeschränkt berücksichtigen. Dies gilt insbesondere in Umgebungen mit eigenen, internen Zertifizierungsstellen, wie in unserem Laboraufbau. Selbst wenn eine CRL technisch korrekt erstellt und öffentlich bereitgestellt wird, heißt das nicht, dass alle Browser diese Informationen automatisch abrufen oder zur Validierung heranziehen.

\subsubsection{Mozilla Firefox}
\input{chapters/06-Revokation/CRL_im_Browser/Firefox.tex}

\subsubsection{Google Chrome}
\subsubsection*{Einleitung}

In einer X.509-basierten Public-Key-Infrastruktur dienen Certificate Revocation Lists (CRLs) und OCSP-Abfragen dazu, den Widerrufsstatus eines Zertifikats zu überprüfen. Obwohl diese Informationen im Zertifikat eindeutig hinterlegt sind – etwa über den CRL Distribution Point (CDP) oder die Authority Information Access (AIA) – werden sie von modernen Browsern sehr unterschiedlich gehandhabt.

Viele Browser greifen diese klassischen PKI-Mechanismen entweder nur eingeschränkt oder gar nicht ab und setzen stattdessen auf eigene, teils proprietäre Widerrufssysteme, die sich meist ausschließlich auf öffentliche Zertifizierungsstellen beziehen. Für interne PKI-Umgebungen, wie in unserem Laboraufbau, bedeutet dies, dass selbst korrekt erstellte und veröffentlichte CRLs nicht zuverlässig ausgewertet werden und Widerrufe aus privaten CAs von den meisten Browsern ignoriert werden.

\subsubsection*{Verhalten von Google Chrome}

Chrome verwendet weder die im Zertifikat eingetragenen CRL-URLs noch führt der Browser OCSP-Abfragen basierend auf dem AIA-Feld durch. Diese Entscheidung ist bewusst getroffen und wird von Google ausführlich begründet. Die wichtigsten Punkte sind:

\begin{itemize}
    \item \textbf{Performance}: Jede OCSP-Abfrage würde eine Netzwerkoperation erfordern und den Seitenaufbau erheblich verzögern.
    \item \textbf{Ausfallsicherheit}: CRLs sind teils mehrere Megabyte groß und würden beim Abruf regelmäßig die Verbindung verlangsamen.
    \item \textbf{Datenschutz}: OCSP-Anfragen verraten dem OCSP-Server jede aufgerufene Website eines Nutzers.
\end{itemize}

Statt CRLs oder OCSP nutzt Chrome ein alternatives Widerrufsmodell: die sogenannten \textit{CRLSets}. Dabei handelt es sich um komprimierte Listen widerrufener Zertifikate, die von Google zentral gepflegt und im Rahmen der regulären Browser-Updates automatisch an alle Chrome-Installationen verteilt werden.

\paragraph*{CRLSets}

CRLSets erfüllen einen grundlegend anderen Zweck als klassische CRLs:

\begin{quote}
    ``CRLSets are designed to provide protection against high-risk revocations, not to represent a complete revocation list.'' \\
    (Chromium Security Team)
\end{quote}

\subparagraph*{Funktionsweise}

\begin{itemize}
    \item Google crawlt öffentliche Widerrufsinformationen ausgewählter CAs.
    \item Nur sicherheitskritische Widerrufe (z.~B. kompromittierte Root- oder Intermediate-Zertifikate) werden in CRLSets aufgenommen.
    \item CRLSets werden als kleines, hochkomprimiertes Paket an Chrome-Nutzer verteilt.
    \item Die im Zertifikat eingebetteten CDP- oder OCSP-Informationen werden ignoriert.
\end{itemize}

\subparagraph*{Einschränkungen}

CRLSets sind ausdrücklich \textbf{nicht} dazu geeignet, individuelle Serverzertifikate zu widerrufen:

\begin{itemize}
    \item End-Entity-Zertifikate werden in der Regel \emph{nicht} aufgenommen.
    \item Private CAs, wie in unserem Laboraufbau, werden vollständig ignoriert.
    \item CRLSets enthalten nur wenige, global relevante Sperrungen.
\end{itemize}

\begin{quote}
    ``CRLSets are not intended to cover every certificate. They are only a stopgap for critical incidents.'' \\
    (Adam Langley, Google)
\end{quote}

\subsubsection*{Problemstellung: Private CA und Widerruf in Chrome}

Da Chrome weder CRLs noch OCSP nutzt und private CAs nicht in das CRLSet-System integriert, ergeben sich im Laboraufbau mehrere praktische Probleme:

\subparagraph*{Chrome ruft die veröffentlichte CRL nicht ab}
Obwohl der CRL Distribution Point korrekt gesetzt ist und die Liste über HTTP abrufbar wäre, berücksichtigt Chrome diese Information nicht.

\subparagraph*{OCSP wird nicht verwendet}
Selbst wenn ein eigener OCSP-Responder betrieben wird, stellt Chrome keine Abfragen an diesen Dienst.

\subparagraph*{CRLSets ignorieren private CAs}
Da Google CRLSets nur für Zertifikate erstellt, die in öffentlichen Root-Stores enthalten sind, werden intern ausgestellte Zertifikate grundsätzlich nicht abgedeckt.

\subparagraph*{Widerruf wird nicht erkannt}
Ein widerrufenes Zertifikat wird in Chrome weiterhin als gültig angezeigt, da der Browser keinerlei Widerrufsmechanismen unserer PKI nutzt.



\subsubsection{Microsoft Edge}
\input{chapters/06-Revokation/CRL_im_Browser/Edge.tex}

\subsubsection{Apple Safari}
\input{chapters/06-Revokation/CRL_im_Browser/Safari.tex}

\subsubsection*{Allgemeine Herausforderungen und Problematik bei Zertifikatswiderrufen}
\input{chapters/06-Revokation/CRL-Problematik/crlproblem.tex}


% ==========================================================
%           **NEUER HAUPTPUNKT: OCSP**
% ==========================================================
\newpage

\section{OCSP – Online Certificate Status Protocol}
\subsection{OCSP in Edge}

\input{chapters/06-Revokation/OCSP_im_Browser/Edge.tex}



